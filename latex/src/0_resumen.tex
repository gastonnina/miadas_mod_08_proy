\newpage
\begin{center}
  \textbf{Resumen}
\end{center}

\noindent El presente informe desarrolla un modelo supervisado para predecir la adopción
de herramientas de Inteligencia Artificial por desarrolladores profesionales,
utilizando el dataset público \textit{Stack Overflow Annual Developer Survey 2025}.
El estudio aplica técnicas de análisis exploratorio, modelamiento y validación
utilizando regresión logística y Random Forest, evaluados mediante F1-score
con validación cruzada k-fold.

Los resultados muestran que la regresión logística sin balance alcanza el mejor
equilibrio entre rendimiento, estabilidad e interpretabilidad (F1=0.8817).
Se presentan además limitaciones, implicaciones prácticas y recomendaciones
para organizaciones tecnológicas en procesos de adopción de IA.

\bigskip
\noindent\textit{Palabras clave:} clasificación supervisada, letras de canciones, TF–IDF, éxito musical, PLN, aprendizaje automático.

\newpage
\begin{center}
  \textbf{Abstract}
\end{center}

\noindent This monograph presents the development of a supervised classification model designed to identify linguistic patterns in Spanish song lyrics associated with musical success, as defined by awards granted by the \textit{Recording Industry Association of America} (RIAA). Natural language processing techniques such as text cleaning, lemmatization, and TF–IDF vectorization were applied to transform lyrics into structured numerical representations. Five machine learning algorithms were evaluated: \textit{Naive Bayes}, \textit{Logistic Regression}, \textit{Support Vector Machines}, \textit{Random Forest}, and \textit{XGBoost}. Model performance was assessed using accuracy, precision, recall, and F1–score. The results provide empirical evidence that lyrical content can serve as a predictor of award recognition, contributing to automated analysis of artistic works and the development of intelligent cultural recommendation systems.

\bigskip
\noindent\textit{Keywords:} supervised classification, song lyrics, TF–IDF, musical success, NLP, machine learning.
