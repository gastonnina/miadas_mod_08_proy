\newpage
\thispagestyle{empty}
\vspace*{0.35\textheight}
\begin{center}
	{\Huge\textbf{CAPÍTULO I}} \\[0.5cm]
	{\Huge\textbf{MARCO INTRODUCTORIO}}
\end{center}
\newpage
\thispagestyle{fancy}

\newpage
\pagenumbering{arabic}
\section*{}
\begin{center}
	\Large \textbf{INTRODUCCIÓN}
\end{center}

Dummy text.

\chapter{MARCO INTRODUCTORIO}

\section{Antecedentes}

Dummy text.



\section{Planteamiento del problema}

Dummy text.

\section{Formulación del problema de investigación}

\vspace{1em}
\noindent
\textit{¿Qué patrones lingüísticos diferencian las letras de canciones en español que han sido premiadas por la RIAA de aquellas que no han sido premiadas durante el período 1990--2022, y cómo pueden ser identificados mediante un modelo de clasificación supervisado?}
\vspace{1em}

\section{Delimitación}

Dummy text.

\section{Problemas específicos}

Dummy text.

\section{Objetivos de la investigación}
    \subsection{Objetivo general}

    \noindent
    Dummy text.

    \subsection{Objetivos específicos}

    Dummy text.

\section{Hipótesis}

\noindent
\textbf{Hipótesis principal:} \textit{Existen patrones lingüísticos diferenciables entre las letras de canciones en español que han sido premiadas por la RIAA y aquellas que no lo han sido, los cuales pueden ser identificados mediante un modelo de clasificación supervisado con un nivel de precisión estadísticamente significativo.}

\subsection{Hipótesis específicas}

Dummy text.

\begin{enumerate}
  \item \textbf{Corpus balanceado.}  
  El corpus recolectado presentará una proporción 1:1 de canciones premiadas y no premiadas por la RIAA, manteniendo una distribución representativa de géneros musicales y periodos temporales (1990–2022).

  \item \textbf{Preprocesamiento y vectorización.}  
  El uso combinado de limpieza textual, normalización, eliminación de stopwords, lematización y extracción de n-gramas generará vectores TF-IDF con mayor capacidad discriminativa (medida en varianza explicada en un análisis de componentes principales) que una simple vectorización unigram.

  \item \textbf{Comparación de modelos.}  
  El modelo XGBoost alcanzará un F1-score significativamente superior (p < 0.05) al de Naive Bayes en el conjunto de prueba, demostrando la ventaja del boosting en datos textuales con desequilibrio de clases.

  \item \textbf{Patrones lingüísticos.}  
  Las variables afectivas (palabras con carga emocional) y los n-gramas de orden dos y tres serán los indicadores más influyentes para distinguir letras premiadas de no premiadas, según la importancia de características extraída de Random Forest y Regresión Logística.

  \item \textbf{Visualización de resultados.}  
  Las nubes de palabras y los mapas de calor mostrarán agrupaciones temáticas claramente diferenciadas entre canciones premiadas y no premiadas, facilitando la interpretación de los patrones lingüísticos identificados.
\end{enumerate}

\section{Justificación}

\subsection{Justificación teórica}

Dummy text.

\subsection{Justificación práctica y social}

Dummy text.

\subsection{Justificación metodológica}

Dummy text.

\section{Alcances y limitaciones}

\subsection{Alcances}

Dummy text.

\subsection{Limitaciones}

Dummy text.


\section{Resultados esperados}

Dummy text.

\section{Matriz de Consistencia}

Dummy text.

\newcolumntype{Y}{>{\raggedright\arraybackslash}X}

\begin{table}[H]
\centering
\caption{Matriz de Consistencia}
\begin{tabularx}{\textwidth}{|Y|Y|Y|Y|}
\hline
\textbf{Problema} & \textbf{Objetivo} & \textbf{Hipótesis} & \textbf{Variables} \\
\hline
\textbf{General:} ¿Qué patrones lingüísticos diferencian las letras de canciones en español premiadas por la RIAA de aquellas que no han sido premiadas durante el período 1990--2022, y cómo pueden ser identificados mediante un modelo de clasificación supervisado?

\vspace{0.3em}
\textbf{Específicos:}
\begin{enumerate}
  \item ¿Cómo construir un corpus balanceado?
  \item ¿Qué técnicas PLN son adecuadas?
  \item ¿Qué modelos clasifican mejor?
  \item ¿Qué características son más relevantes?
  \item ¿Cómo representar visualmente los resultados?
\end{enumerate}
&

\textbf{General:} Desarrollar un modelo de clasificación supervisado que permita identificar y analizar patrones lingüísticos característicos de las letras de canciones en español premiadas y no premiadas por la RIAA.

\vspace{0.3em}
\textbf{Específicos:}
\begin{enumerate}
  \item Recolectar un corpus balanceado
  \item Preprocesar con técnicas de PLN
  \item Entrenar y comparar modelos
  \item Identificar características lingüísticas
  \item Visualizar patrones relevantes
\end{enumerate}
&

\textbf{General:} Existen patrones lingüísticos diferenciables entre letras premiadas y no premiadas que pueden ser detectados por un modelo con precisión significativa.

\vspace{0.3em}
\textbf{Específicas:}
\begin{itemize}
  \item El contenido lingüístico varía entre clases
  \item Algunos modelos son más precisos
  \item Las visualizaciones ayudan a interpretar resultados
\end{itemize}
&

\textbf{Independiente:} Contenido lingüístico de las letras

\vspace{0.3em}
\textbf{Dependiente:} Éxito musical (premiación)

\vspace{0.3em}
\textbf{Auxiliares:} Desempeño del modelo, visualización e interpretabilidad
\\
\hline
\end{tabularx}
\label{tab:matriz_consistencia}
\end{table}